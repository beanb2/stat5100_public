\documentclass[11pt]{article}

\usepackage[utf8]{inputenc} % allow utf-8 input
\usepackage[T1]{fontenc}    % use 8-bit T1 fonts
\usepackage{hyperref}       % hyperlinks
\usepackage{url}            % simple URL typesetting
\usepackage{amsfonts}       % blackboard math symbols
\usepackage[margin=1in]{geometry} % Sets paper margins
\usepackage{natbib} % Makes references look nice.


\title{Predicting Design Ground Snow Loads in Utah}


\author{(Your name here)}

\begin{document}
\maketitle



\section{Introduction}

\begin{itemize}
\item Introduce the design ground snow load problem and why it is important. You may choose to cite a few sources regarding the importance of proper design ground snow loads such as the reference provided in the assignment description \citep{Arcement2017}
\end{itemize}

\section{Data}
\begin{itemize}
\item Briefly describe the variables of interest, including their units of measure. 
\item Provide one or two exploratory plots that show the distribution of the data as well as the relationship between ground snow and elevation. 
\item Identify any outliers that are present in the data. If you do identify outliers, be sure to describe how you handled them in the following section. 
\end{itemize}

\section{Modeling Assumptions}
\begin{itemize}
\item Determine whether the original regression model provided in the starter code satisfies the necessary modeling assumptions. 
\item If assumptions are violated, describe the remedial measures you took to eliminate the violations. 
\item Demonstrate that your remedial measures were successful. 
\end{itemize}

\section{Model Inference and Validation}
\begin{itemize}
\item Determine whether or not the linear relationship between ground snow and elevation is significant. 
\item Provide an interpretation for the coefficient associated with elevation in the model. 
\item Provide simultaneous prediction intervals for Park City (elevation: 2134 meters) and Logan (elevation: 1382 meters) and comment on the appropriateness of these intervals. 
\item Provide a confidence interval for your prediction at Kings Peak, Utah (elevation 4122 meters) and comment on the appropriateness of this interval. 
\item On a scale of 1-5 (1 being terrible, 5 being excellent) rate the quality of your linear model and justify your choice using two graphical and two numerical outputs. 
\end{itemize}

\section{Conclusions}
\begin{itemize}
\item Describe the implications of your analysis: what do we learn about the relationship between ground snow and elevation based on your work?
\item Identify at least two future directions for research. What additional questions would be worth exploring based on your analysis? 
\item Be sure to write the introduction and conclusion in ways that an everyday reader would understand. These sections should not be overly technical. 
\end{itemize}

\bibliographystyle{apalike}  
\bibliography{project1_references}

\appendix
\section{Appendix}
Place relevant SAS code here:

\begin{verbatim}
/* This first line of code will need to be changed */
FILENAME REFFILE '/home/u41171697/data/project1/snowloads.csv';

/* Read in the csv file using proc import. Note that you will need
   to upload the snowloads.csv file to SAS Studio prior to use */ 
PROC IMPORT DATAFILE=REFFILE replace
	DBMS=CSV
	OUT=WORK.snow;
	GETNAMES=YES;
RUN;

/* Initital regression model */
proc reg data = snow;
model snowload = elevation;
run;

\end{verbatim}

\end{document}