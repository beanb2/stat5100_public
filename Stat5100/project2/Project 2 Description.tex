\documentclass[12pt]{report}
\parindent=0pt
\pagestyle{empty} \oddsidemargin  -0.25in \evensidemargin -0.25in
\textwidth      6.5in \headheight     -0.25in \topmargin      0.0in
\textheight=9.5in
\usepackage{ulem}

\setlength{\parskip}{-.7ex}

\begin{document}

\Large
\begin{center}
   \textbf{Project 2}
   
\vspace{0.5cm}
   
 Stat5100
  
\end{center}
\normalsize

\vspace{1em}

Notable Due Dates:

\vspace{1em}
\begin{tabular}{r l l}
{\bf Report (First Draft):} & Nov 6 \\
{\bf Peer Review:} & Nov 13 \\
{\bf Proof of Writing Fellow Review:} & Nov 19 \\
{\bf Report (Final Draft):} & Nov 20 \\
\end{tabular}

\vspace{1em}

This project is intended to give you experience designing and implementing regression models on a real dataset while formally communicating your results in writing. Unlike Project 1, you are not required to follow a specific paper template.

\vspace{1em}

\large{\bf Data:}\\
\normalsize

Tinder is a mobile dating app that allows users to locate potential dates in their local area.
A 2017 publication, ``Where Dating Meets Data: Investigating
 Social and Institutional Privacy Concerns in Tinder'' in {\it Social Media + Society}, reported on results from a survey of
 nearly 500 US Tinder users.  The authors of that paper, Lutz and Ranzini, have graciously made their data available
for use in this STAT 5100 course (but you do {\it not} need to look at that paper to do this project).
These data have been organized in the tinder.xlsx file on Canvas.\\

Responses to the authors' questionnaire have been combined within several categories of questions
to give composite scores on some of the variables in the table on the following page.
Unless coding levels are defined in the table, variables
are coded on an increasing scale, so that higher values mean ``more''.\\

Of interest in this project is whether and
how various demographic, psychological, motivational, and opinion characteristics
affect (or at least predict) how genuine a person's self-presentation is on Tinder.\\

\vspace{1em}

\large {\bf General Directions:}
\normalsize

\vspace{1em}

You will use the multiple linear regression methods discussed
in this course to analyze the data, creating a valid linear model to
predict a Tinder user's level of self-presentation genuineness.  Randomly withhold 10\% of the data as a test set for validation and report the requested MSPRs outlined later in this description. \\


\newpage 

\large {\bf Columns of File tinder.xlsx}
\normalsize

\vspace{1em}


\begin{center}
\begin{tabular}{l l}
Variable Name & Definition\\ \hline
ID & user identifier (arbitrary and meaningless other than to identify specific users)\\ \hline
Genuine & in terms of how they present themselves on Tinder, how genuine is the user\\
  & (i.e., how honest and realistic, as opposed to fake or falsified or made-up)\\ \hline
SocPrivConc & how concerned is the user that other users will mis-use their private data\\ \hline
InstPrivConc & how concerned is the user that Tinder will mis-use their private data\\ \hline
Narcissism & how narcissistic is the user\\  \hline
SelfEsteem & how much self-esteem does the user have\\ \hline
Loneliness & how lonely is the user\\ \hline
Hookup & how interested is the user in using Tinder to hook up (especially for sex)\\ \hline
Friends & how interested is the user in using Tinder to build friendships\\ \hline
Partner & how interested is the user in using Tinder to develop a partnership\\ \hline
Travel & how interested is the user in using Tinder while traveling\\ \hline
SelfValidation & how interested is the user in using Tinder for self-validation\\ \hline
Entertainment & how interested is the user in using Tinder for entertainment\\ \hline
Orientation & user's sexual orientation (1=heterosexual, 2=homosexual, 3=bisexual,\\
&  4=other)\\  \hline
Gender & user's identified gender (1=male, 2=female, 3=other)\\ \hline
Education & user's education level (1=no schooling, 2=high school graduate,\\
          & 3=some college, 4=undergrad degree, 5=masters degree,\\
          &  6=doctoral degree, 7=other)\\ \hline
Income & user's estimated level of income (1=low, 2=medium, 3=high, 4=unknown)\\ \hline
Employment & user's current employment status (1=employed, 2=sef-employed,\\
           & 3=out of work and looking, 4=out of work but not looking, 5=homemaker,\\
           & 6=student, 7=military, 8=retired, 9=unable to work)\\ \hline
Age & user's age in years\\ \hline
ImpFitness & how important does the user think it is for members of their same gender\\
          & to have physical fitness\\ \hline
ImpEnergy & how important does the user think it is for members of their same gender\\
          & to have energy (or stamina)\\ \hline
ImpAttractive & how important does the user think it is for members of their same gender\\
          & to have physical attractiveness\\ \hline
\end{tabular}
\end{center}

\newpage

\large {\bf Specific Directions:}\\
\normalsize

Your paper should include a discussion of the following elements:\\

\begin{itemize}
\item Describe the nature and background of the data and provide a motivation for the analysis (for a hypothetical audience). 
\item Discuss the approaches you used to build your regression model, including, but not limited to:
\begin{itemize}
\item Variable transformations (must be attempted if model fails to meet assumptions)
\item Variable selection 
\item Interaction terms
\item Satisfaction/Violation of assumptions regarding residuals. 
\end{itemize}
\item Report your final model equation and provide a few interpretations of some of the statistically significant coefficients in your model. 
\item Report the validation error (using the test set) and compare this to the validation error for a reduced model which contains only an intercept. 
\item Place the results of your study in context. Discuss the implications of your results for the general tinder user and identify potential future directions for research. 
\item Include the code you used in your analysis, with appropriate code comments, in the appendix. 
\end{itemize}


The report should include clear graphical displays with appropriate
explanations and interpretations, discussions of model assumptions,
justifications of decisions made in the analysis (such as which
variables to include in the model, which transformations to make,
and how to deal with influential observations or outliers), and interpretations of model
components. Figures and Tables should be included in the main body
of the report and referenced in the text, per the homework style guide.  Do not include any unnecessary computer output or
code (i.e., the appendix should include only the code used to
generate the results and plots referred to in the report).  The length of the final report should be between 6-10 pages, not including the title page and appendix.\\

Please be sure to check out a copy of the assignment rubric in canvas before submitting either draft of your paper. 

\end{document}
