\documentclass{article}\usepackage[]{graphicx}\usepackage[]{color}
% maxwidth is the original width if it is less than linewidth
% otherwise use linewidth (to make sure the graphics do not exceed the margin)
\makeatletter
\def\maxwidth{ %
  \ifdim\Gin@nat@width>\linewidth
    \linewidth
  \else
    \Gin@nat@width
  \fi
}
\makeatother

\definecolor{fgcolor}{rgb}{0.345, 0.345, 0.345}
\newcommand{\hlnum}[1]{\textcolor[rgb]{0.686,0.059,0.569}{#1}}%
\newcommand{\hlstr}[1]{\textcolor[rgb]{0.192,0.494,0.8}{#1}}%
\newcommand{\hlcom}[1]{\textcolor[rgb]{0.678,0.584,0.686}{\textit{#1}}}%
\newcommand{\hlopt}[1]{\textcolor[rgb]{0,0,0}{#1}}%
\newcommand{\hlstd}[1]{\textcolor[rgb]{0.345,0.345,0.345}{#1}}%
\newcommand{\hlkwa}[1]{\textcolor[rgb]{0.161,0.373,0.58}{\textbf{#1}}}%
\newcommand{\hlkwb}[1]{\textcolor[rgb]{0.69,0.353,0.396}{#1}}%
\newcommand{\hlkwc}[1]{\textcolor[rgb]{0.333,0.667,0.333}{#1}}%
\newcommand{\hlkwd}[1]{\textcolor[rgb]{0.737,0.353,0.396}{\textbf{#1}}}%
\let\hlipl\hlkwb

\usepackage{framed}
\makeatletter
\newenvironment{kframe}{%
 \def\at@end@of@kframe{}%
 \ifinner\ifhmode%
  \def\at@end@of@kframe{\end{minipage}}%
  \begin{minipage}{\columnwidth}%
 \fi\fi%
 \def\FrameCommand##1{\hskip\@totalleftmargin \hskip-\fboxsep
 \colorbox{shadecolor}{##1}\hskip-\fboxsep
     % There is no \\@totalrightmargin, so:
     \hskip-\linewidth \hskip-\@totalleftmargin \hskip\columnwidth}%
 \MakeFramed {\advance\hsize-\width
   \@totalleftmargin\z@ \linewidth\hsize
   \@setminipage}}%
 {\par\unskip\endMakeFramed%
 \at@end@of@kframe}
\makeatother

\definecolor{shadecolor}{rgb}{.97, .97, .97}
\definecolor{messagecolor}{rgb}{0, 0, 0}
\definecolor{warningcolor}{rgb}{1, 0, 1}
\definecolor{errorcolor}{rgb}{1, 0, 0}
\newenvironment{knitrout}{}{} % an empty environment to be redefined in TeX

\usepackage{alltt}

\usepackage{float}

% Set the margins on the page to not be so large
\addtolength{\oddsidemargin}{-.875in}
\addtolength{\evensidemargin}{-.875in}
\addtolength{\textwidth}{1.75in}
\addtolength{\topmargin}{-.875in}
\addtolength{\textheight}{1.75in}

% Take off page numbering
\pagenumbering{gobble}
\IfFileExists{upquote.sty}{\usepackage{upquote}}{}
\begin{document}

\title{%
  1.4: Data Exploration \\
  \smallskip
  \large Stat 5100: Dr. Bean
}
\date{}

\maketitle

\section{Why Data Exploration}

Data Modeling is a lot like:

\begin{figure}[H]
\centering
\includegraphics[width=0.5\textwidth]{../figures/module1/bridgeJumping.jpg}
\end{figure}

In order to avoid disaster, you need to \textbf{look} before you \textbf{jump}.

Example:
Consider four scenarios where we use to create a model that uses values of $x$ to predict values of $y$. We make the assumption in each case that the data can be modeled as
\begin{equation}
Y_i = \beta_0 + \beta_1X_{i,1} + \epsilon_i
\end{equation}

This assumption means that we assume that $X$ and $Y$ share a linear relationship. That is, as $X$ increases, $Y$ will increase proportionally. We will explore this further in Handout 2.1.

Data Explorations BEFORE modeling will help us to detect:
\begin{itemize}
\item Skewed distributions
\item Outlier points
\item Non-linear trends
\end{itemize}

Often, we can use \textbf{variable transformations} to get data that are normal, or at least symmetric, in distribution.

\textbf{Common Exploratory Plots}
\begin{itemize}
\item \textbf{Boxplots:}: Show the five quartlies of the data (min, 25th percentile, median, 75th percentile, and maximum).
\begin{itemize}
\item Values that are farther than 1.5*IQR (Interquartile Range, which is the 75th percentile minus the 25 percentile) above the 75th percentile or below the 25th percentile are typically plotted as ``outlier'' points.
\item Great way to quickly summarize the range of values.

\end{itemize}

\end{document}
