\documentclass[12pt]{notes}

% Command for Questions
%\question{}

% Command for Notes
% \note{}

% Code to create a minipage where you can type in class notes. 
%%\begin{minipage}[l][2cm][c]{\textwidth}
%\begin{comment}

%\end{comment}
%%\end{minipage}

\usepackage{listings}

% In order for the minted code to run, we had to create a new compilation routine called pdflatex+shellEscape.
% This includes a --shell-escape command which should ONLY be used when pygmentized is required as it compromises security. 
% We also had to add pygmentize (a python package) to the system path (BEFORE miktex) and then restart the computer. 
\usepackage{minted}
\usemintedstyle{borland}
\lstset{language=SAS, 
  breaklines=true,  
  basicstyle=\ttfamily\bfseries,
  columns=fixed,
  keepspaces=true,
  identifierstyle=\color{blue}\ttfamily,
  keywordstyle=\color{cyan}\ttfamily,
  stringstyle=\color{purple}\ttfamily,
  commentstyle=\color{green}\ttfamily,
  } 
  
% \begin{minted}{sas}
% \end{minted}


% Begin Document
%==============================================================================
\begin{document}
% Include the Title of the Handout
\ntitle{2.3: Simple Model Inference}

Recall the simple linear model
\[
Y_i = \beta_0 + \beta_1X_{i,1} + \epsilon_i. 
\]

Inference is the process by which we make a decision about whether an observed difference from an expectation was simply due to chance or not. 

\nspace
In other words, inference is the process of making conclusions given incomplete information. 

% Include Numbered Sections
\section{Why Inference?}

Hypothetical questions:
\begin{itemize}
\item Suppose you found out that there is not significant relationship between study time and final grades in Stat 5100, how would this effect your approach to this course?
\item Suppose you have the flu and you find out from a clinical trial of a new flu drug that those who took the drug had slightly shorter flu durations than those who took the placebo, but that the difference was likely due to chance. How likely would you be to purchase this drug? 
\end{itemize}

\nspace
In the absence of complete information, inference is an efficient way to decide what associations are ``real'' and which are not.

\question{(Individual) Can you think of an example scenario where a test of significance would be of interest to researchers?}


\section{Hypothesis Testing}
Recall that hypothesis testing is the formal way by which we determine if an observed difference from an expectation was due to chance. 

\nspace
\subsection*{Process}
\bi
\item Define a null and alternative hypothesis. 
\bi
\item $H_0:$ ``no effect''
\item $H_a:$ ``some effect''
\ei
\item Define a test statistic:
\bi
\item Compares what we observed to what we expected if the null hypothesis was true. 
\ei
\item Determine the ``sampling distribution''
\bi
\item Determines the natural variation in the test statistic that we would expect if we took many different samples from the same population. 
\item In practice, we only ever take one sample. Statistical theory is what allows us to determine what the distribution would look like if we could take many samples. 
\item The distribution often relies on \textbf{model assumptions}. 
\ei
\item Get p-value
\bi
\item This is the probability of obtaining an observation as far, or farther, away from what we expected if the null was true. 
\ei
\item Make conclusion in context.
\bi
\item If the p-value is small ($<\alpha$), then it is unlikely that we would have obtained our observation if the null hypothesis is true. This provides evidence that the observed difference between our observation and expectation is REAL, and not simply due to chance. 
\ei
\ei

\subsection{Toluca Example:}
\textbf{If model assumptions are satisfied}, then $b_1 \sim N(0, \sigma^2\{b_1\}$.  

\begin{minipage}[l][2cm][c]{\textwidth}
%\begin{comment}
\note{$\sim$ means ``follows'' while $\sigma^2\{b_1\}$ represents the true variance of $b_1$, as estimated by $s\{b_1\}$.}
%\end{comment}
\end{minipage}

\nspace
Our test statistic then becomes
\[
t = \frac{b_1 - 0}{s\{b_1\}} \sim t_{df_E} = 10.29
\]
with $25-2 = 23$ degrees of freedom with a \textbf{p-value $<$ 0.0001}.

\begin{minipage}[l][3cm][c]{\textwidth}
%\begin{comment}
\note{where $df_E$ is the degrees of freedom for the residuals, which is $n-2$ in the simple linear model case}

\nspace 

\note{draw t-distribution and shade the area that represents the p-value}
%\end{comment}
\end{minipage}

Since our p-value is lower than our level of significance (which is typically 0.05 and something we set beforehand), we would \textbf{reject} the null hypothesis \textbf{and conclude} that there is significant evidence that lot size and work hours are linearly related. 

\subsection*{Where did $\alpha = 0.05$ come from?}
Short answer: Sir Ronald Fisher, a prominent statistician, made it up:

\begin{quotation}
It is a common practice to judge a result significant, if it is such a magnitude that it would have been produced by chance not more frequently than once in twenty trials. This is an arbitrary, but convenient, level of significance for the practical investigator...\footnote{As on p99 of ``The Lady Tasting Tea'' (2001) by David Salsburg. See \url{http://jse.amstat.org/v16n2/velleman.pdf} for more discussion about statistical theory.}
\end{quotation}

However, $\alpha = 0.05$ has proven to be a good level of significance that balances the probability of Type I (claiming a difference when there isn't one) and Type II (claiming \textit{no} difference when there \textit{is} one). 


\subsection*{Consider the following}

You wish to determine if Aggie ice cream is more fattening than other ice cream shops in Logan. Suppose your null hypothesis is: ``Aggie ice cream has the same number of calories per cup as Charlie's ice cream.'' You then conduct a test and obtain a p-value of 0.048, indicating that there is evidence that the average caloric counts are significantly different. You then realize that you forgot to include 5 sample in your study. When you include these additional samples, you obtain a p-value of 0.052, indicating no significant difference. 

\nspace
\question{(Groups) What will be your final conclusion based on this information?}

\begin{minipage}[l][3cm][c]{\textwidth}
%\begin{comment}
\note{It depends. What is the cost of claiming a difference when there isn't one, versus the cost of claiming no difference when there is one?}
%\end{comment}
\end{minipage}

\section*{Confidence Intervals}


















% End the Document
%==============================================================================
\end{document}