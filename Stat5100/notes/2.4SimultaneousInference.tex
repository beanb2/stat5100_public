\documentclass[12pt]{notes}

\usepackage{hyperref}

% Command for Questions
%\question{}

% Command for Notes
% \note{}

% Code to create a minipage where you can type in class notes. 
%%\begin{minipage}[l][2cm][c]{\textwidth}
%\begin{comment}

%\end{comment}
%%\end{minipage}


% Begin Document
%==============================================================================
\begin{document}
% Include the Title of the Handout
\ntitle{2.4: Simultaneous Inference and Important Considerations}
Simultaneous inference is when we want to conduct multiple tests of significance at the same time.

% Include Numbered Sections
\section{Why Simultaneous Inference?}
In handout 2.3, we conducted inference for parameters one at a time. We need to change our approach when looking at multiple parameters simultaneously. 

\question{(Groups) How and why do we need to change our approach when conducting simultaneous inference?}

(check out \href{https://xkcd.com/882/}{\textbf{\underline{this comic}}} for help).

\begin{minipage}[l][3cm][c]{\textwidth}
%\begin{comment}
\note{If we conduct several tests at the same level of significance, the probability of getting one false positive result (a type I) error becomes much higher than $\alpha$.}

\nspace
\note{As a result, we need to adjust the level of significance to account for a ``multiplicity'' of testing.}
%\end{comment}
\end{minipage}

\section{Bonferroni Adjustment}
Multiplicity:
\bi
\item Let $A_j =$ event that an individual $(1-\alpha)100\%$ CI does not contain the true value of $\beta_j.$
\item $P(A_0) = P(A_1) = \alpha \rightarrow$ Type I Error
\bi
\item $P(\text{NOT} A_j) = $ probability that an interval contains the true value of $\beta_j$.
\ei
\item \textbf{Bonferroni Inequality:} $P(\text{NOT} A_0 \text{ AND NOT } A_1) \ge 1 - P(A_0) - P(A_1)$ 
\ei

This means that if we conduct $g$ tests at a confidence level of $(1 - \frac{\alpha}{g})$, then we are guaranteed that overall level of confidence for all intervals \textit{considered jointly} will be at least $(1-\alpha)$, we call this the \textbf{Bonferroni adjustment}. 

\nspace
\bi
\item \textbf{Bonferroni Advantage:} Can be literally applied in \textit{any} situation requires a multiplicity adjustment, including simultaneous intervals for $\hat{Y}$ at multiple $X_h$ levels. 
\item \textbf{Bonferroni Disadvantage:} Can be overly conservative, producing inefficient (unnecessarily wide) intervals. 
\ei

\subsection*{Comparison of Simultaneous Intervals for $\hat{Y}$}
\bi
\item Confidence intervals (mean response)
\bi
\item Bonferroni
\[
\hat{Y} \pm t_{n-p}(1-\frac{\alpha}{2g})*s\{\hat{Y}_h\}
\]
\item Working-Hotelling (WH)
\[
\hat{Y} \pm W*s\{\hat{Y}_h\} \qquad \left(W = \sqrt{pF_{p, n-p}(1-\alpha)}\right)
\]

\begin{minipage}[l][2cm][c]{\textwidth}
%\begin{comment}
\note{Notice that the W-statistic does not consider $g$}
%\end{comment}
\end{minipage}

\bi
\item WH provides a ``confidence band'' for the entire regression line (all possible $X_h$ levels). 
\item This means the WH interval at any individual $X_h$ will be wider than the t-based confidence interval, but the WH intervals will eventually be narrower than Bonferroni confidence intervals if enough $X_h$ are considered. 
\ei
\ei

\item Prediction intervals (new response)

\bi
\item Bonferroni
\[
\hat{Y} \pm t_{n-p}(1-\frac{\alpha}{2g})*s\{\hat{Y}_{h (new)}\}
\]
\item Scheffe (chef-eh)
\[
\hat{Y} \pm S*s\{\hat{Y}_{h (new)}\} \qquad \left(S = \sqrt{gF_{g, n-p}(1-\alpha)}\right)
\]
\ei
\ei

\textbf{Rule of Thumb: Always pick the most efficient interval that guarantees your intended type I error ($\alpha$)}. 




















% End the Document
%==============================================================================
\end{document}