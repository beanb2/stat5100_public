\documentclass[12pt]{../notes}

% Command for Questions
%\question{}

% Command for Notes
% \note{}

% Code to create a minipage where you can type in class notes. 
%%\begin{minipage}[l][2cm][c]{\textwidth}
\begin{comment}

\end{comment}
%%\end{minipage}



% Begin Document
%==============================================================================
\begin{document}
% Include the Title of the Handout
\ntitle{4.1: Penalized Regression}

% Include Numbered Sections
\section{Why Penalized Regression?}


\nspace 
\question{What are some undesirable consequences of having estimates of $\beta_k$'s with inflated variance?}

 \begin{minipage}[l][5cm][c]{\textwidth}
\begin{comment}
\note{
\bi
\item Interpretation: the sign/magnitude of the estimated coefficients could be misleading or non-intuitive
\item Stability: Coefficients could change drastically for small changes in the training data, which makes it hard to persuade others that the model form is correct. 
\item Variable selection: When the number of candidate explanatory variables is large, inflated variance may cause us to throw the ``best'' predictor variables out in a stepwise search. 
\ei
}
\end{comment}
\end{minipage}

\nspace 

\question{Why is it critical that we \textbf{standardize} our variables prior to using any of the penalized regression techniques?}

 \begin{minipage}[l][2cm][c]{\textwidth}
\begin{comment}
\note{The penalty terms do not respect differences in the \textbf{scale} of variables. Variables with a small range of values will be unfairly punished if we do not standardize.}
\end{comment}
\end{minipage}


\nspace
  
\question{Which of the following is NOT a good scenario to used penalized regression techniques? Why?
\begin{enumerate}
\item Facebook is trying to create a model to predict the likelihood of a user responding positively to a certain type of ad. 
\item The Huntsman Cancer institute is trying to determine which active genes in a person’s DNA increase the likelihood of Pancreatic cancer. 
\item The USU Agriculture Experiment Station is trying to determine if a change in the composition of feed significantly influences the milk output of dairy cows.
\end{enumerate}
}

\begin{minipage}[l][4cm][c]{\textwidth}
\begin{comment}
\note{
\textbf{3} is the correct answer because:
\bi 
\item This scenario is an experiment rather than an observational study. 
\item We are interested in the significance of an effect, rather than accurate predictions. 
\ei
}
\end{comment}
\end{minipage}

\newpage
\question{Which method does NOT get estimated coefficients exactly equal to zero as the penalty parameter increases? Why?
\begin{itemize}
\item Ridge Regression
\item LASSO 
\item Elastic Net
\end{itemize}}

\begin{minipage}[l][3cm][c]{\textwidth}
\begin{comment}
\note{Ridge Regression: The use of the squared penalty term makes it nearly impossible for coefficients to converge to be exactly equal to zero as the penalty parameter increases.}
\end{comment}
\end{minipage}

\question{Given the following output, determine the value of the intercept for the following ridge regression model.}

\begin{figure}[H]
\centering
\includegraphics[width=\textwidth]{../figures/module4/ridge_regression_intercept.png}
\end{figure}

\begin{minipage}[l][3cm][c]{\textwidth}
\begin{comment}
\note{$$95.423 - 1.315*7.462 - 0.306*48.154 + 0.129*11.769 + 0.343*30 = 82.684$$}
\end{comment}
\end{minipage}




% End the Document
%==============================================================================
\end{document}