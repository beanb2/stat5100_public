\documentclass[12pt]{../notes}

% Command for Questions
%\question{}

% Command for Notes
% \note{}

% Code to create a minipage where you can type in class notes. 
%%\begin{minipage}[l][2cm][c]{\textwidth}
\begin{comment}

\end{comment}
%%\end{minipage}
  
% \begin{minted}{sas}
% \end{minted}


% Begin Document
%==============================================================================
\begin{document}
% Include the Title of the Handout
\ntitle{4.2: Variations on OLS (Ordinary Least Squares)}

\question{If Robust Regression is less sensitive to outliers and influential points, why should we \textit{ever} use ordinary least squares (OLS) regression and worry about outliers and influential points?}

\begin{minipage}[l][5cm][c]{\textwidth}
\begin{comment}
% Example:
\note{IF assumptions regarding residuals are satisfied, then OLS parameter estimates truly are the ``best'' (i.e. unbiased with minimum variance). Robust regression involves the (somewhat arbitrary) selection of a weighting function, which affects the estimates and the statistical inference of those estimates. Robust regression is also computationally more expensive. In short, if a simpler solution is equally effective in solving the problem, we should always prefer the simpler solution.
\vspace{0.25cm}
\newline
Metaphor: You can technically use a sledgehammer to pound in a nail, but it will likely be a little messy. Just because a sledgehammer can ``do the job'' of a regular hammer does not diminish the value of owning a hammer. 
}  
\end{comment}
\end{minipage}

\nspace
\question{In what ways is non-linear regression different from simply adding higher ordered predictors (i.e. $X^2$ or $X^3$) in our model?}

\begin{minipage}[l][3cm][c]{\textwidth}
\begin{comment}
\note{Higher ordered predictors are still linear for the beta coefficients. Thus higher ordered terms can only describe a limited number of non-linear relationships. We use non-linear regression when the domain theory suggests that the model form should be non-linear in the betas.}
\end{comment}
\end{minipage}

\nspace
\question{Why would we want to avoid ``making up'' or own non-linear model forms for non-linear regression? Why depend so much on the domain theory for the model form?}

\begin{minipage}[l][2cm][c]{\textwidth}
\begin{comment}
\note{We want the non-linear model to have physical meaning. While our predictions from the made up model might look OK, the estimated parameters will likely have no meaningful interpretation.}
\end{comment}
\end{minipage}

\nspace
\question{What are some potential issues with the use of gradient descent to find optimal model coefficients?}

\begin{minipage}[l][4cm][c]{\textwidth}
\begin{comment}
\note{\begin{itemize}
\item speed: It can take a long time to reach the optimal solution when you are taking small steps at each iteration. 
\item local extremes: The gradient descent cannot recognize if the solution it converged to is the global minimizer of the loss function, or just a \textit{local} minimizer. 
\end{itemize}}
\end{comment}
\end{minipage}

\newpage
\question{The residuals of your OLS regression model show signs of heteroskedasticity. You are able to identify a transformation for Y that fixes the issue. Should you use OLS with the transformed Y or use weighted least squares regression? Please give arguments in favor of both methods.}

\begin{minipage}[l][4cm][c]{\textwidth}
\begin{comment}
\note{\begin{itemize}
\item Argument for OLS: simpler model form, ability to conduct traditional statistical inference. Easier to compute. 
\item Argument for weighted least squares: avoid the use of a transformation, which makes model coefficients easier to interpret. 
\end{itemize}}
\end{comment}
\end{minipage}






% End the Document
%==============================================================================
\end{document}