\documentclass{article}\usepackage[]{graphicx}\usepackage[]{color}
% maxwidth is the original width if it is less than linewidth
% otherwise use linewidth (to make sure the graphics do not exceed the margin)
\makeatletter
\def\maxwidth{ %
  \ifdim\Gin@nat@width>\linewidth
    \linewidth
  \else
    \Gin@nat@width
  \fi
}
\makeatother

\definecolor{fgcolor}{rgb}{0.345, 0.345, 0.345}
\newcommand{\hlnum}[1]{\textcolor[rgb]{0.686,0.059,0.569}{#1}}%
\newcommand{\hlstr}[1]{\textcolor[rgb]{0.192,0.494,0.8}{#1}}%
\newcommand{\hlcom}[1]{\textcolor[rgb]{0.678,0.584,0.686}{\textit{#1}}}%
\newcommand{\hlopt}[1]{\textcolor[rgb]{0,0,0}{#1}}%
\newcommand{\hlstd}[1]{\textcolor[rgb]{0.345,0.345,0.345}{#1}}%
\newcommand{\hlkwa}[1]{\textcolor[rgb]{0.161,0.373,0.58}{\textbf{#1}}}%
\newcommand{\hlkwb}[1]{\textcolor[rgb]{0.69,0.353,0.396}{#1}}%
\newcommand{\hlkwc}[1]{\textcolor[rgb]{0.333,0.667,0.333}{#1}}%
\newcommand{\hlkwd}[1]{\textcolor[rgb]{0.737,0.353,0.396}{\textbf{#1}}}%
\let\hlipl\hlkwb

\usepackage{framed}
\makeatletter
\newenvironment{kframe}{%
 \def\at@end@of@kframe{}%
 \ifinner\ifhmode%
  \def\at@end@of@kframe{\end{minipage}}%
  \begin{minipage}{\columnwidth}%
 \fi\fi%
 \def\FrameCommand##1{\hskip\@totalleftmargin \hskip-\fboxsep
 \colorbox{shadecolor}{##1}\hskip-\fboxsep
     % There is no \\@totalrightmargin, so:
     \hskip-\linewidth \hskip-\@totalleftmargin \hskip\columnwidth}%
 \MakeFramed {\advance\hsize-\width
   \@totalleftmargin\z@ \linewidth\hsize
   \@setminipage}}%
 {\par\unskip\endMakeFramed%
 \at@end@of@kframe}
\makeatother

\definecolor{shadecolor}{rgb}{.97, .97, .97}
\definecolor{messagecolor}{rgb}{0, 0, 0}
\definecolor{warningcolor}{rgb}{1, 0, 1}
\definecolor{errorcolor}{rgb}{1, 0, 0}
\newenvironment{knitrout}{}{} % an empty environment to be redefined in TeX

\usepackage{alltt}

\usepackage{float}

% Set the margins on the page to not be so large
\addtolength{\oddsidemargin}{-.875in}
\addtolength{\evensidemargin}{-.875in}
\addtolength{\textwidth}{1.75in}
\addtolength{\topmargin}{-.875in}
\addtolength{\textheight}{1.75in}

% Take off page numbering
\pagenumbering{gobble}
\IfFileExists{upquote.sty}{\usepackage{upquote}}{}
\begin{document}

\title{%
  3.4.1: R - Model Validation \\
  \smallskip
  \large Stat 5100: Dr. Bean
}
\date{}

\maketitle

Project 2 is focused on using information regarding Tinder profiles
to predict the genuineness of the user. Information regarding the total
set of variables are included in the project 2 description. For purposes
of illustration, only a subset of variables are considered here.

\begin{knitrout}
\definecolor{shadecolor}{rgb}{0.969, 0.969, 0.969}\color{fgcolor}\begin{kframe}
\begin{alltt}
\hlcom{# Set a seed so that results are reproducible}
\hlkwd{set.seed}\hlstd{(}\hlnum{222}\hlstd{)}
\hlkwd{library}\hlstd{(stat5100)}

\hlcom{# If you run this code, you will have to import the data differently. This}
\hlcom{# particular dataset is not included in the stat5100 package for copyright}
\hlcom{# reasons, hence why I load it locally here.}
\hlstd{tinder} \hlkwb{<-} \hlkwd{read.csv}\hlstd{(}\hlstr{"../../../../data_csv/tinder.csv"}\hlstd{,} \hlkwc{na.strings} \hlstd{=} \hlstr{"."}\hlstd{,}
                   \hlkwc{stringsAsFactors} \hlstd{=} \hlnum{FALSE}\hlstd{)}

\hlcom{# All the column names in the tinder dataset}
\hlkwd{names}\hlstd{(tinder)}
\end{alltt}
\begin{verbatim}
##  [1] "ID"             "Genuine"        "SocPrivConc"    "InstPrivConc"  
##  [5] "Narcissism"     "SelfEsteem"     "Loneliness"     "Hookup"        
##  [9] "Friends"        "Partner"        "Travel"         "SelfValidation"
## [13] "Entertainment"  "Orientation"    "Gender"         "Education"     
## [17] "Income"         "Employment"     "Age"            "ImpFitness"    
## [21] "ImpEnergy"      "ImpAttractive"
\end{verbatim}
\begin{alltt}
\hlkwd{apply}\hlstd{(tinder,} \hlnum{2}\hlstd{, class)}
\end{alltt}
\begin{verbatim}
##             ID        Genuine    SocPrivConc   InstPrivConc     Narcissism 
##    "character"    "character"    "character"    "character"    "character" 
##     SelfEsteem     Loneliness         Hookup        Friends        Partner 
##    "character"    "character"    "character"    "character"    "character" 
##         Travel SelfValidation  Entertainment    Orientation         Gender 
##    "character"    "character"    "character"    "character"    "character" 
##      Education         Income     Employment            Age     ImpFitness 
##    "character"    "character"    "character"    "character"    "character" 
##      ImpEnergy  ImpAttractive 
##    "character"    "character"
\end{verbatim}
\begin{alltt}
\hlcom{# These columns by default are set as factors, which are categorical variables.}
\hlcom{# To make this work out for a continuous response variable, we need to convert}
\hlcom{# some of the columns to numeric values.}
\hlstd{numeric_columns} \hlkwb{<-} \hlkwd{c}\hlstd{(}\hlstr{"Genuine"}\hlstd{,} \hlstr{"SocPrivConc"}\hlstd{,} \hlstr{"InstPrivConc"}\hlstd{,} \hlstr{"Narcissism"}\hlstd{,}
                     \hlstr{"SelfEsteem"}\hlstd{,} \hlstr{"Loneliness"}\hlstd{,} \hlstr{"Hookup"}\hlstd{,} \hlstr{"Friends"}\hlstd{,} \hlstr{"Partner"}\hlstd{,}
                     \hlstr{"Travel"}\hlstd{,} \hlstr{"SelfValidation"}\hlstd{,} \hlstr{"Entertainment"}\hlstd{,} \hlstr{"Age"}\hlstd{,}
                     \hlstr{"ImpFitness"}\hlstd{,} \hlstr{"ImpEnergy"}\hlstd{,} \hlstr{"ImpAttractive"}\hlstd{)}
\hlkwa{for} \hlstd{(nc} \hlkwa{in} \hlstd{numeric_columns) \{}
  \hlstd{tinder[[nc]]} \hlkwb{<-} \hlkwd{as.numeric}\hlstd{(tinder[[nc]])}
\hlstd{\}}

\hlcom{# Some of the variables we want to be factors (for example, different gender}
\hlcom{# classes, sexual orientations, etc.)}
\hlstd{factor_columns} \hlkwb{<-} \hlkwd{c}\hlstd{(}\hlstr{"Orientation"}\hlstd{,} \hlstr{"Gender"}\hlstd{,} \hlstr{"Education"}\hlstd{,} \hlstr{"Income"}\hlstd{,} \hlstr{"Employment"}\hlstd{)}
\hlkwa{for} \hlstd{(fc} \hlkwa{in} \hlstd{factor_columns) \{}
  \hlstd{tinder[[fc]]} \hlkwb{<-} \hlkwd{as.factor}\hlstd{(tinder[[fc]])}
\hlstd{\}}

\hlcom{# Separate the data into training and test sets. Here we will withhold 20%}
\hlcom{# for validation.}
\hlstd{n} \hlkwb{<-} \hlkwd{nrow}\hlstd{(tinder)}
\hlstd{training_index} \hlkwb{<-} \hlkwd{sample}\hlstd{(}\hlnum{1}\hlopt{:}\hlstd{n,} \hlkwc{size} \hlstd{=} \hlnum{0.20} \hlopt{*} \hlstd{n)}

\hlstd{tinder_train} \hlkwb{<-} \hlstd{tinder[training_index, ]}
\hlstd{tinder_test} \hlkwb{<-} \hlstd{tinder[}\hlopt{-}\hlstd{training_index, ]}

\hlcom{# Fit one model with 4 variables}
\hlstd{tinder_lm1} \hlkwb{<-} \hlkwd{lm}\hlstd{(Genuine} \hlopt{~} \hlstd{SocPrivConc} \hlopt{+} \hlstd{InstPrivConc} \hlopt{+} \hlstd{Narcissism} \hlopt{+} \hlstd{SelfEsteem,}
                 \hlkwc{data} \hlstd{= tinder_train)}

\hlcom{# Fit another model with more variables}
\hlstd{tinder_lm2} \hlkwb{<-} \hlkwd{lm}\hlstd{(Genuine} \hlopt{~} \hlstd{SocPrivConc} \hlopt{+} \hlstd{InstPrivConc} \hlopt{+} \hlstd{Narcissism} \hlopt{+}
                   \hlstd{SelfEsteem} \hlopt{+} \hlstd{Loneliness} \hlopt{+} \hlstd{Hookup} \hlopt{+} \hlstd{Friends} \hlopt{+} \hlstd{Partner} \hlopt{+}
                   \hlstd{Travel} \hlopt{+} \hlstd{SelfValidation} \hlopt{+} \hlstd{Entertainment,} \hlkwc{data} \hlstd{= tinder_train)}

\hlcom{# To fit a third model with no predictors, we simply use the average of the}
\hlcom{# response variable (Genuine). Having this third "model" can help us decide}
\hlcom{# if there is any significant improvement using the predictors over simply}
\hlcom{# guessing the average.}
\hlstd{tinder_lm3} \hlkwb{<-} \hlkwd{lm}\hlstd{(Genuine} \hlopt{~} \hlnum{1}\hlstd{,} \hlkwc{data} \hlstd{= tinder_train)}
\end{alltt}
\end{kframe}
\end{knitrout}

\subsubsection*{Calculate MSPR for each model}

\begin{knitrout}
\definecolor{shadecolor}{rgb}{0.969, 0.969, 0.969}\color{fgcolor}\begin{kframe}
\begin{alltt}
\hlcom{# To do this, we make predictions with the testing dataset and then compare}
\hlcom{# it to the known value of the response variable Genuine in the testing dataset.}
\hlstd{tinder_test_pred1} \hlkwb{<-} \hlkwd{predict}\hlstd{(tinder_lm1,} \hlkwc{newdata} \hlstd{= tinder_test)}
\hlstd{tinder_test_pred2} \hlkwb{<-} \hlkwd{predict}\hlstd{(tinder_lm2,} \hlkwc{newdata} \hlstd{= tinder_test)}
\hlstd{tinder_test_pred3} \hlkwb{<-} \hlkwd{predict}\hlstd{(tinder_lm3,} \hlkwc{newdata} \hlstd{= tinder_test)}

\hlstd{mspr1} \hlkwb{<-} \hlkwd{mean}\hlstd{((tinder_test_pred1} \hlopt{-} \hlstd{tinder_test}\hlopt{$}\hlstd{Genuine)}\hlopt{^}\hlnum{2}\hlstd{,} \hlkwc{na.rm} \hlstd{= T)}
\hlstd{mspr2} \hlkwb{<-} \hlkwd{mean}\hlstd{((tinder_test_pred2} \hlopt{-} \hlstd{tinder_test}\hlopt{$}\hlstd{Genuine)}\hlopt{^}\hlnum{2}\hlstd{,} \hlkwc{na.rm} \hlstd{= T)}
\hlstd{mspr3} \hlkwb{<-} \hlkwd{mean}\hlstd{((tinder_test_pred3} \hlopt{-} \hlstd{tinder_test}\hlopt{$}\hlstd{Genuine)}\hlopt{^}\hlnum{2}\hlstd{,} \hlkwc{na.rm} \hlstd{= T)}

\hlcom{# Show results}
\hlkwd{data.frame}\hlstd{(mspr1, mspr2, mspr3)}
\end{alltt}
\begin{verbatim}
##      mspr1    mspr2    mspr3
## 1 2.701384 2.329362 2.747672
\end{verbatim}
\begin{alltt}
\hlcom{# Based upon the MSPR, it looks like models 1 and 3 perform roughly the same}
\hlcom{# on the testing dataset, but model 2 does better.}
\end{alltt}
\end{kframe}
\end{knitrout}


\end{document}
