\documentclass[12pt]{notes}

\begin{document}
\ntitle{Handout 1: Introduction to Modern Regression Methods}

\section{About me}

\begin{figure}
\includegraphics[0.6\textwidth]{}
\end{figure}

\begin{itemize}
\item Graduated from BYU-Idaho in 2014 with a Bachelors of Science in Applied Mathematics.
\item Graduated from Utah State University in 2019 with a PhD in Mathematical Sciences.
\item Current interests include basketball, cross country skiing, hiking and spending time with my wife and daughter.
\end{itemize}


\question{(Groups) What is a creative, yet appropriate, question that you have about the life/career of the instructor?}

\begin{minipage}[l][2cm][c]{\textwidth}

\end{minipage}

\section{Why Modern Regression Methods?}

Statistics, in the words of Dr. Bin Yu, is the ``science that solves data problems.'' This science becomes more and more relevant in a world inundated with data. From the late Leo Breiman:

\begin{quotation}
The uses of statistics pervade our society. They are used and terribly misused all through the social sciences and health fields. ... It is surprising how much the world around us depends on the use of statistics. ... It’s odd that even though the articles involving statistics in the newspapers far outnumber those involving say, physics or chemistry, people in general know very little about what we do.
\end{quotation}

In this class, we will learn several of the foundational approaches for using data to make \textbf{predictions}. Perhaps more importantly, we will discuss the \textbf{cautions} we must consider when using and interpreting model output. 

\question{(Groups) Why are YOU taking this course?}

\begin{minipage}[l][3cm][c]{\textwidth}
\end{minipage}


\section{Functional vs Statistical Modeling}

We learn about functions in Math 1050 (College Algebra), a functional model takes a set of inputs $X$ and produces a (set of) outputs $Y$, i.e.
\[Y = f(X)\]

Example: 
You write a function to model the profits from your lemonade stand. You rent the stand for \$200 a month and sell each glass of lemonade for \$1.00. If it costs you \$0.25 to make the lemonade then your monthly profits $Y$ could be modeled as a function of the number of lemonade glasses you sell $x$
\[Y = 0.75x - 200\]. 

The key to a functional model is that each input $x$ produces a \textbf{unique} output $Y$.

\vspace{0.5cm}
In a \textbf{statistical model} we assume that the values of $Y$ can be modeled by a function \textit{plus} some ``random noise'' $\epsilon$. The presence of the $\epsilon$ term allows for many different values of $Y$ for the same set in inputs $x$. 






















\end{document}